\chapter{Language $\mathcal{AOPL}$}

\subsubsection{Checking Compliance}

To evaluate the compliance of happening $h$ at state $s$ with an obligation policy $P$, \citet{gelfond_authorization_2008} extend $lp$ as follows:

\begin{gather}
    lp(obl(h)) =_{def}
        obl(h). \\
    lp(\neg obl(h)) =_{def}
        \neg obl(h). \\
    lp(obl(h) \textbf{ if } F) =_{def}
        obl(h) \leftarrow
            lp(F). \\
    lp(\neg obl(h) \textbf{ if } F) =_{def}
        \neg obl(h) \leftarrow
            lp(F). \\
    lp(d_i: \textbf{normally } obl(h) \textbf{ if } F_i) =_{def}
        obl(h) \leftarrow
            lp(F),
            \textbf{not } ab(d),
            \textbf{not } \neg obl(h). \\
    lp(d_j: \textbf{normally } \neg obl(h) \textbf{ if } F_j) =_{def}
        \neg obl(h) \leftarrow
            lp(F),
            \textbf{not } ab(d),
            \textbf{not } obl(h). \\
    lp(prefer(d_i, d_j)) =_{def}
        ab(d_j) \leftarrow lp(F_1).
\end{gather}

\begin{definition}
    $obl(h) \in P(s)$ iff the logic program $lp(P, s)$ entails $obl(h)$
\end{definition}

\begin{definition}
    \label{def:obligation_event_compliance}
    An event $<s, A>$ is \textit{compliant} with an obligation policy $P$ if $(\forall obl(a) \in P(s) \ a \in A) \land (\forall obl(\neg a) \in P(s) \ a \not \in A)$
\end{definition}

This definition of compliance contrasts with that for authorization policies.
Given a set of actions $A$, the definition for authorization policies can be thought of first enumerating action $a \in A$ then checking to see if $permitted(a)$ exists.
The definition for obligation policies, by contrast, can be thought of first searching for $obl(a)$ statements, then checking to see if $a \in A$ exists (or $a \not \in A$ in the case of $obl(\neg a)$).

Furthermore, the definition for obligation compliance does not differentiate between strong compliance and weak compliance.
Were such a differentiation to exist, it would center around instances where neither $\neg obl(a)$ nor $obl(a)$ is true\footnotemark.
For such a case, an obligation policy would not explicitly impose nor explicitly waive an agent from an obligation to perform action $a$.
However, since $\mathcal{AOPL}$ does not consider `weak compliance', this situation is simply called \textit{compliant}.

\footnotetext{
    A similar statement can be made about when neither $\neg obl(\neg a)$ nor $obl(\neg a)$ is true.
}

\begin{definition}
    Let $P$ be an arbitrary policy $P$ written in $\mathcal{AOPL}$.

    \begin{itemize}
        \item Let $P_a$ be a derivative policy of $P$ that only has the authorization statements of $P$.
            $P_a$ is an \textit{authorization policy induced by} $P$~\citep{gelfond_authorization_2008}.
        \item Likewise, let $P_o$ be a derivative of $P$ such that $P_o$ only has the obligation statements of $P$.
            $P_o$ is an \textit{obligation policy induced by} $P$~\citep{gelfond_authorization_2008}.
    \end{itemize}
\end{definition}

\begin{definition}
    Let $P$ be an arbitrary policy $P$ written in $\mathcal{AOPL}$.
    Let $P_a$ and $P_o$ be the authorization and obligation policy induced by $P$, respectively.

    \begin{itemize}
        \item An event $<s, A>$ is \textit{strongly compliant} with $P$ if $<s, A>$ is strongly compliant with $P_a$ and if $<s, A>$ is compliant with $P_o$~\citep{gelfond_authorization_2008}.
        \item An event $<s, A>$ is \textit{weakly compliant} with $P$ if $<s, A>$ is weakly compliant with $P_a$ and if $<s, A>$ is compliant with $P_o$~\citep{gelfond_authorization_2008}.
        \item An event $<s, A>$ is \textit{non-compliant} with $P$ if $<s, A>$ is neither strongly compliant nor weakly compliant with $P$\footnotemark.
    \end{itemize}
\end{definition}

\footnotetext{
    This definition is not explicitly provided by \citet{gelfond_authorization_2008}.
}
