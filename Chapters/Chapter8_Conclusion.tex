\chapter{Conclusion}

% Present the following:
%
% \begin{itemize}
%     \item summary of your work
%     \item list of contributions
%     \item limitations and drawbacks to your approach
%     \item future work
%     \item other important concluding details
% \end{itemize}
%
% Also, here's the Miami Logo to demonstrate including images in your thesis.
% The logo is seen in Figure~\ref{fig:MiamiLogo}.
%
% \begin{figure}[h]
%     \centering
%     % \includegraphics[width=0.4\textwidth]{Figures/Miami}
%     \caption{The Miami University Logo}
%     \label{fig:MiamiLogo}
% \end{figure}
%
% \section{Future Work}

In this paper, we have surveyed different agent architectures ranging from production rule systems, logic-based architectures (such as BDI architectures), and those with normative models and cognitive models.
The focus of this work has been on modern logic-based architectures.
We began by introducing transitions systems, which model the agent’s environment like a directed graph of states.
Then we discussed the development of a formal syntax that specifies transition systems.
Next, we introduced the AAA agent architecture.
While the AAA architecture is goal-oriented and uses information encoded in an action language to achieve them, it does not persist a plan across iterations of its control loop.
This is deemed a limitation and prompts the creation of the AIA architecture.

The AIA architecture represents an agent’s mental state as properties of the world that are internal to the agent.
The AIA architecture then uses this mental state to persist an agent’s plans (which are encapsulated in a construct called an activity) as well as its current progress in executing them.
The agent’s mental state is maintained by a set of logic rules called the Theory of Intentions ($\mathcal{TI}$).
We then provided the AIA control loop and explained its execution through the aid of a few examples.

To conclude, there has been significant research towards designing different agent architectures.
This paper has focused on the AIA agent architecture and we plan to build upon it in future work.
