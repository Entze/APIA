\chapter{Introduction}

% Use this chapter to introduce readers to your research.
%
% Discuss motivations, contributions and other introductory material.
%
% \textbf{Note:}
% In general, the suggestions provided in this template (for content, chapter breakdown, etc.) are just that - suggestions.
% Consult with your advisor to determine the appropriate thesis structure for your sub-discipline, and adjust accordingly.

Computers have increasingly become a ubiquitous part of everyday life.
As more intelligence is able to be embedded in them, we are starting to see the rise in autonomous systems.
Automatic stock trading programs and self-driving cars are but a few examples.
However, as we expect them to perform more complex tasks, the question arises of how to best construct agents in a computer program that are capable of reasoning over higher-level concepts.

This paper provides a background on different \textit{agent architectures} that have been proposed in prior literature.
This work then narrows its focus to discuss a particular class of agent architectures that are heavily based in the formal theory of logic.
It first introduces preliminary definitions on which they are based, then moves on to discuss them in significant detail with the aid of examples.

\section{Motivation}

\section{Contributions}
