\chapter{Background \& Related Work}

% This section is where you will discuss relevant background work, and related works for comparison.
% Ensure that you cite references appropriately, using this as an example~\cite{sample2019}
%
% \section{Background Topic 1}
%
% \section{Background Topic 2}
%
% \section{Related Work}

\section{Agents}

There has been prior research and discussion to determine what constitutes an ``agent''.
Though work in \cite{wooldridge_agent_1995} fails completely define the term,
the authors do propose a set of characteristics that agents must have.
They are summarized in \cite{dignum_intentional_1998} and are as follows.

First, the agent must have autonomy to complete its own actions without the intervention of human.
This implies that it is making decisions on its own power.
Second, the choices of an agent must be best explained in terms of some form of \textit{intention}.
These intentions can be desires, goals, etc.
An implication of this requirement is that agents cannot be explained through low-level concepts.
Lastly, an agent explained with particular intentions should seem to actually progress the satisfaction of its intentions.
For example, a system with the intention of achieving a goal,
but fails to do so when alternatives would have done so, cannot be called an ``agent''.

There are many different kinds of agents.
\cite{balke_how_2014} provides a survey of five different kinds of agents.

The first of which are production rule systems \cite{balke_how_2014}.
Production rule systems are symbolic in nature and
consist of three core components:

\begin{itemize}

    \item A set of \textit{rules} (also called \textit{productions}).
        Each rule has the form $ C_i \rightarrow A_i $,
        where $ C_i $ is what is called the \textit{sensory precondition}
        and $ A_i $ is the action to be performed if $ C_i $ is true.
        $ C_i $ is essentially an ``if'' condition and $ A_i $ is the conditionally executed body.

    \item A knowledge base that stores domain-relevant information about the agent's environment.
        This is called the \textit{working memory}.

    \item A \textit{rule interpreter} determines which rules apply for the current state of the working memory.
        In the case of rule conflicts, it decides which rule should be executed.

\end{itemize}

% TODO: Consider talking about architecture

Production rule systems are usually implemented in Prolog or LISP \cite{balke_how_2014}.

The next class of agents are those ascribing to the BDI architecture \cite{balke_how_2014}.
The belief-desire-intention (BDI) model combines a philosophical model of human practical reasoning
with several successful applications \cite{georgeff_belief-desire-intention_1999}.
BDI agents are built on the idea of mental states,
consisting of three items: beliefs, desires, and intentions \cite{balke_how_2014}.
Beliefs refer to an agent's representation of the world.
An agent's beliefs something to be true if and only if the statement is found in its representation.
The beliefs can be inconsistent with the actual state of the environment -- the agent can be wrong.

\cite{rao_modeling_1991}

Desires are all possible courses of action that an agent might want to accomplish in order to reach a goal \cite{balke_how_2014}.
An agent is not necessarily committed to its desires.
They just play a part in its decision process.
Intentions, on the other hand, are commitments to particular courses of action to reach a goal.
Courses of action of an intention is called a plan.

In addition, BDI agents have a library of plans.
This library consists of pre-computed primitive logic rules to signify which actions contribute to accomplishing which goals.
At each reasoning step, BDI agents search through the plan library for plans to see which plans have a post-condition that matches the currently selected intention.
It sorts plans according to relevance.

The third class of agents are those conform to normative models.
They differ from BDI agents in that they are externally motivated.
BDI agents are internally motivated since all major components (beliefs, goals, and intentions) are internal to the agent.
Norms imposed on normative agents are created by the agent’s environment.
