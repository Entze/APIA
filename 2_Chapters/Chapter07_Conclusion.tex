\chapter{Conclusion}

% Present the following:
%
% \begin{itemize}
%     \item summary of your work
%     \item list of contributions
%     \item limitations and drawbacks to your approach
%     \item future work
%     \item other important concluding details
% \end{itemize}
%
% Also, here's the Miami Logo to demonstrate including images in your thesis.
% The logo is seen in Figure~\ref{fig:MiamiLogo}.
%
% \begin{figure}[h]
%     \centering
%     % \includegraphics[width=0.4\textwidth]{Figures/Miami}
%     \caption{The Miami University Logo}
%     \label{fig:MiamiLogo}
% \end{figure}
%
% \section{Future Work}

In this thesis, we have surveyed different agent architectures ranging from production rule systems, logic-based architectures (such as BDI architectures), and those with normative models and cognitive models.
The focus of this work has been on modern logic-based architectures.
We began by introducing transitions systems, which model the agent's environment like a directed graph of states.
Then we discussed the development of a formal syntax that specifies transition systems.
Next, we introduced the AAA agent architecture.
While the AAA architecture is goal-oriented and uses information encoded in an action language to achieve them, it does not persist a plan across iterations of its control loop.
This is deemed a limitation and prompts the creation of the $\mathcal{AIA}$ architecture.

The $\mathcal{AIA}$ architecture represents an agent's mental state as properties of the world that are internal to the agent.
The $\mathcal{AIA}$ architecture then uses this mental state to persist an agent's plans (which are encapsulated in a construct called an activity) as well as its current progress in executing them.
The agent's mental state is maintained by a set of logic rules called the Theory of Intentions ($\mathcal{TI}$).
We then provided the $\mathcal{AIA}$ control loop and explained its execution through the aid of a few examples.

We introduced languages $\mathcal{APL}$ and $\mathcal{AOPL}$ to represent and reason over action compliance given authorization and obligation policies.
While $\mathcal{APL}$ and $\mathcal{AOPL}$ are not action languages, they share a similar syntax to that of action language $\mathcal{AL}$ and work in tandem with it.
We demonstrated the implementation of $\mathcal{APL}$ and $\mathcal{AOPL}$ using \textsc{clingo} with a varying degree of knowledge, ranging from a full knowledge of fluents to an initial understanding of fluents coupled with an event history.

We then contrasted the $\mathcal{AIA}$ agent architecture and $\mathcal{AOPL}$ to the PDC-agent architecture.
One feature of the PDC-agent architecture is that it supports multi-agent cooperation.
PDC-agent also shares a similar mechanism to $\mathcal{TI}$ for updating beliefs via internal actions.
However, the PDC-agent architecture requires enumerating plans by hand and annotating which goals they achieve.
Since the $\mathcal{AIA}$ architecture is based on action languages, plans can be automatically generated, and their respective end-goal can be computed.

We extend the $\mathcal{AIA}$ architecture so that it can reason over $\mathcal{AOPL}$ policies.
This new agent architecture will be called the $\mathcal{APIA}$ architecture.
In addition to its formal definition and implementation, we construct examples on which $\mathcal{APIA}$ can be evaluated.
The evaluation of $\mathcal{APIA}$ will also consist of the expressivity of its syntax, its elaboration tolerance, its difficulty of construction, and its runtime performance.
Due to performance improvements in underlying ASP solvers, we update Blount's implementation of the $\mathcal{AIA}$ architecture from CR-Models2 to \textsc{clingo} so that differences in performance are meaningful.

The $\mathcal{APIA}$ agent architecture is a modern logic-based architecture that is able to plan amid changing environments while maintaining policy compliance.
As a result, its behavior is both explainable and ensured to be compliant.
With such properties, the $\mathcal{APIA}$ architecture is a significant contribution to the logic programming community.
